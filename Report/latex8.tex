
%
%  $Description: Author guidelines and sample document in LaTeX 2.09$ 
%
%  $Author: ienne $
%  $Date: 1995/09/15 15:20:59 $
%  $Revision: 1.4 $
%

\documentclass[times, 10pt,twocolumn]{article} 
\usepackage{latex8}
\usepackage{times}

%\documentstyle[times,art10,twocolumn,latex8]{article}

%------------------------------------------------------------------------- 
% take the % away on next line to produce the final camera-ready version 
\pagestyle{empty}

%------------------------------------------------------------------------- 
\begin{document}

\title{Design and Implementation of Distributed Application: Project Report}

\author{
Wallace Garbim\\Instituto Superior Tecnico\\wallace.garbim@tecnico.ulisboa.pt
\and
Miguel Crespo\\Instituto Superior Tecnico\\miguel.crespo@tecnico.ulisboa.pt
\and
Florian Ehrenstorfer\\
Instituto Superior Tecnico\\fehrenstorfer@gmail.com\\
% For a paper whose authors are all at the same institution, 
% omit the following lines up until the closing ``}''.
% Additional authors and addresses can be added with ``\and'', 
% just like the second author.
}

\maketitle
\thispagestyle{empty}

\begin{abstract}
In this paper we present a simplified function-as-a-service cloud platform, which enables the distributed execution of a chain of operators on a shared storage. 
The system consists of one scheduler and multiple worker and storage nodes.
Workload is balanced equally between workers through the scheduler.
Data consistency and fault tolerance are guaranteed by the storage nodes.
We present the problem and motivation first, then describe our solution in detail, and evaluate it at last in this paper.
\end{abstract}
%------------------------------------------------------------------------- 
\Section{Introduction}
Distributed function-as-a-service cloud platforms are increasing in popularity with big cloud providers like AWS and Google Cloud offering custom solutions.
They offer great flexibility with of the shelf plug and play options as well as customizable and self-developed options.
For this project we developed our own distributed function-as-a-service cloud platform, a simplified version of other platforms.
We focused on its design, implementation and evaluation in the project in this paper.

%------------------------------------------------------------------------- 
\Section{Problem}
The function-as-a-service cloud platform allows to run applications in a distributed network.
Applications consist of a chain of operators, which can access a shared storage system.
Because the storage system is distributed as well, one of the main problems is data consistency and fault tolerance.
In order to maintain data consistency a gossip protocol is used to propagate updates through the network.
The storage system supports three operations: read, write, and updateIfValueIs.
Records are stored with the value and a version number.
For the identification of the correct storage server to contact about a value, consistent hashing is used.
For data consistency, operators have to read consistent version of the data, which has to be ensured by the system.
Because the system also has to provide fault-tolerance, data has to be distributed between storage nodes, and reads have to be redirected.
The different operators are assigned to workers by a scheduler, and then send from one worker to another after the operator has finished.


%------------------------------------------------------------------------- 
\Section{Solution}
In the following section, we will describe our proposed solution. \newline

Our system consists of five main processes: The scheduler, worker and storage nodes, and the Puppetmaster and PCS processes.
The Puppetmaster is the main entry point of our system and controls the startup and workflow of the whole system.
The PCS enables the Puppetmaster to startup the three different types of nodes remotely.
Because all three nodes receive their configuration information as command line arguments at startup, the whole system could also be run without the Puppetmaster and PCS.
The Puppetmaster can be run with single commands in the command line in a loop or by providing a full script, which will be executed by the Puppetmaster. \newline
To run the Puppetmaster the following command has to be executed inside the Puppetmaster directory:
\emph{dotnet run}. \newline
To run the Puppetmaster with a full script the following command has to be executed inside the Puppetmaster directory: 
\emph{dotnet run < scriptname}.
scriptname has to include the path to the script. \newline
The PCS process can be started with \emph{dotnet run} inside the PCS directory.
It only has to be started once to start multiple other nodes.
All printed output from the started nodes will be printed in this console.
The Puppetmaster sends start up requests to the PCS, which then starts worker, storage or scheduler nodes as a new process.
All other types of requests are also sent from the Puppetmaster to PCS, where they are distributed to the relevant nodes. \newline
The scheduler is responsible for the assignment of operators to workers and the distribution of workload between them.
Workers receive the chain of operators and execute the operator that was assigned to them, before storing the output in the chain and sending the chain to the next worker in line.
Since they only receive the name of the operator, workers select the operator with reflection.
For that the operator .dll file has to be placed inside the \emph{Operator} folder inside the workers directory.
The workers get the url of all storage nodes and the scheduler from the meta data of the chain.
\newline


\subsection{Scheduler Implementation}

The scheduler is responsible for the assignment of operators to workers and the distribution of workload between them.


We first opted for a simple round-robin solution to the scheduling problem, meaning that the operators were assigned one by one, in a round-robin fashion.

On our second implementation, we decided to take into account the amount of work that each worker had done and assigned the next operator to the worker that
had done the least amount of work. Because of this, workers send completion confirmations to the scheduler when they have finish working on one operator, and the scheduler 
keeps track of the amount of time each worker has worker for, resulting in a better load balancing between the available workers.


\subsection{Storage Implementation Overview}
The storage system stores pairs of keys and values, aswell as up to 10 previous versions of the same key. All the keys are
replicated in all of the nodes. The storage supports three types of operations: Read, Write and Update If. The Write operations are
propagated through a gossip protocol, while the Update If operations are executed in all nodes simultaneously using the 2-Phase-Commit
protocol.

When an update is issued, it is assigned a Version Number and Replica Identifier, aswell as some other timestamps relevant for the correct 
functioning of the gossip protocol. In our implementation, every key has a replica timestamp and a value timestamp associated with it. 
The gossip protocol used was the one described on the course's book.

\subsubsection{Read Operations}

Read operations are treated as simple queries to the server. If a read operation is received with a null DIDAVersion, the most recent version is returned,
otherwise, the request version is returned. If the version is not found in the server, an error is returned.

\subsubsection{Write Operations}    
Write operations need to be propagated to the other nodes through the gossip protocol, but an order also needs to be maintained between them, so that Update If operations
can be executed on the same record in every node. For updates that are causally constrained (i.e. sequential updates issued to the same replica), the gossip protocol already takes care of this issue. 
For concurrent updates (i.e. updates that are executed in different replicas at the same time), we opted for simply ordering the updates based on their version number and replica id. Therefore, the most recent update
on a replica is the one with the highest version number and lowest replica ID. For example, if we have three replicas, R1, R2, R3 and they all write concurrently to the same key K, they all generate and update with the
same version number, but with different replica IDs. The update issued by R1 is going to be the most recent, as it has the lowest replica id. When the system stabilizes, all the replicas agree on what is the next version 
number to be given to a new version of that key, so updates with a higher version number will be causally dependent on updates with lower version number.

A replica can't process a write and an Update If at the same time.

\subsubsection{Update If Operations}

Update If operations are applied to the whole system at once, through a 2 Phase Commit protocol. 

On the first phase, when an Update If reaches a replica R, for a key K, it blocks any further updates to that key until the Update If is applied to all replicas. It sends a message to 
every storage node with the last version it has for key K and the new desired value. When the other replicas receive this message, they check if the version they have for that key 
matches the one that was sent. If the versions match, it sends an OK to the coordinator and appends the update to a pending Update If list, waiting for confirmation from the coordinator.
If the versions don't match, it sends back a NOK.

On the second phase, if the coordinator receives any NOK message, it cancels the transaction, sending an abort message to all the storage nodes involved.
If all participants respond with an OK message, the coordinator sends a doCommit message to all the participants and they all execute the update. The update might still fail if the provided old value
does not match the value present in the record.

If two Update Ifs are concurrent (i.e. they are issued on different replicas at the same time) the system will abort both Update If operations, for simplicity reasons.

\subsubsection{Update Propagation}

The gossip protocol used is very close to the one described in the course's book, so we aren't going to go into too much detail about it.

Once a write request reaches the server, it is assigned a version number, replica Identifier and Lamport Clock corresponding to the last update seen by the client that sent the update.
It is also checked if the update has already been seen (through the unique identifier of the update, generated by the client).

If the update is stable, meaning that it doesn't depend on an update that the replica hasn't seen, it is executed immediately. 
This update is then inserted into a log, that is propagated to every node in the system.

Once a node receives a gossip message, it merges the incoming log with its own, filtering any repeated entries, and checks to 
see if any update on its log has become stable. If so, the update is executed.
A table of executed updates is kept, in order to avoid executing duplicate updates.

The clients also keep a Lamport Clock of the latest update they have seen, so that they don't run the risk of reading inconsistent values from a replica that is less up to date.


\subsubsection{Data Consistency}

Data consistency between worker nodes executing the same client script is ensured by storing the last written or read version of a certain record 
in the DIDAMetaRecord and passing this information along the chain of workers.

\subsubsection{Failure Detection}

The system detects failures by pinging the storage nodes periodically. When a storage node does not respond within the stipulated time, it is assumed dead and the 
information is sent to all the other storage nodes. These then remove that node from their list of other known storage servers. The PCS also knows what nodes are dead or alive,
so it won't send information about dead storages to the workers.

\subsubsection{Fault Tolerance}

For N storage nodes, the system support N-1 failures, as there only needs to be one storage server running to accept requests from any worker.
If a replica fails before it can send any new updates that it might have gotten after it sent the last gossip message, these updates are lost.


%------------------------------------------------------------------------- 
\Section{Evaluation}
In this section we will evaluate our solution with a baseline scenario, as well as best and worst cases.

\SubSection{Baseline}
For the Baseline evaluation of our system we use a modified version of the Puppetmaster and operator script provided for the checkpoint evaluation. 
We run four storage nodes and two worker nodes.
The used operators are the UpdateAndChainOperator and the AddOperator.
We compare the performance of our final system with replication, data consistency and fault tolerance, with an earlier version without replication of data.
For that we time how long the whole application runs, before returning to the Puppetmaster.
Because the script includes waits to ensure the correct startup of all nodes, the values should only be used for comparison between the two systems.
The full execution on the old system without data replication took 9825 milliseconds.


%------------------------------------------------------------------------- 

\Section{Further Work}
There are a lot of aspects that could have been bettered in our solution, but due to lack of time we didn't implement these solutions.

\subsection{Update If operations}

In our solution, it would would have been better if two replicas that disagreed with the latest version of a key could communicate with eachother and try to reach
a state where the replica that was lagging behind could update itself and not cancel the entire transaction. The already existing replica Logs used on the gossip protocol could have been
exchanged between the nodes, so that missing updates could be shared and even speedup the propagation speed of said updates.

Other improvement could consist in, instead of aborting two concurrent update if operations, to establish an order between them and execute them in that order.

\subsection{Partial Replication}

We did not implement partial replication on our solution, but it could easily be implemented by dividing the pool of available storage nodes into a set of groups and limiting
the nodes known by any node to the ones on its group. 


%------------------------------------------------------------------------- 
\Section{Conclusion}


%------------------------------------------------------------------------- 
\nocite{ex1,ex2}
\bibliographystyle{latex8}
\bibliography{latex8}

\end{document}

